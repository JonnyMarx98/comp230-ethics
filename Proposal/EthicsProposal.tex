\documentclass{scrartcl}

\usepackage[hidelinks]{hyperref}
\usepackage[none]{hyphenat}

\title{Ethics and Professionalism Essay Proposal}
\subtitle{COMP230 - Ethics and Professionalism Essay}

\author{1603748}

\begin{document}

\maketitle

\section*{Topic}

My essay will be on: "What are the ethical issues of cloning video games and what can be done to prevent the copying of Intellectual Property". The games industry Is an extremely competitive market, and it is often a challenge for developers to think of good game ideas, this can lead to them "stealing" the idea from other developers. This is known as "cloning", they take the core idea of the game, sometimes they change it a little bit and change the look of the game, then they sell it as if it were there own game. I this essay I will talk about the ethical and moral issues that this causes and potential solutions to prevent cloning from happening. Is it wrong to clone a game, should it remain legal as long as you change the aesthetic of the game, or should there be stricter laws introduced to prevent the copy of Intellectual Property. 

% Add details as appropriate.

\section*{Paper 1}
% This is an example! Replace the details with a paper relevant to your chosen topic.
\begin{description}
\item[Title:] The current status of copyright and patent protection for computer software
\item[Citation:] \cite{Current}
\item[Abstract:] ``Copyright law continues to provide the favored protection for software, including programs of all kinds, microcode and the screen imagery of video games and computer interfaces. However, patent protection for nonobvious software inventions is increasingly available as additional or alternative protection in many industrial countries. In countries other than the US, patent protection for computer programs requires a statement of combination with implementing hardware, but the hardware itself must be novel in only a very few countries. The US imposes the most stringent disclosure requirements, requiring an inventor to describe the best mode for carrying out his invention, in effect calling for source code or, at a minimum, a detailed flow diagram. Copyright law, although protecting only 'expression' and not any concept or idea in itself, has thus far been implemented in US court decisions so as to provide a reasonable degree of protection, providing an incentive for innovative work without preventing the creation of new works to perform required computer functions. Recent proposals for specially tailored laws to permit 'reverse engineering' or copying of programs that are deemed industry standards are unnecessary and counterproductive.''
\item[Web link:] \url{http://ieeexplore.ieee.org.ezproxy.falmouth.ac.uk/document/128340/}
\item[Full text link:] \url{http://ieeexplore.ieee.org.ezproxy.falmouth.ac.uk/stamp/stamp.jsp?arnumber=128340}
\item[Comments:] 
\end{description}

\section*{Paper 2}
\begin{description}
\item[Title:] Near-miss software clones in open source games: An empirical study
\item[Citation:] \cite{NearMiss}
\item[Abstract:] Developers tend to reuse source code by copy/paste. This form of reuse introduces code clones to software systems. Cloning in games can happen in different levels of granularity. The extreme case is known as Game Clone where the complete project is being cloned, e.g., by making a new independent branch which the original game's source code constitutes the seed for the new branch. Although it has been more than two decades since research on code clones started, various characteristics of cloning in open source games has not been studied. Therefore, there is no specific evidence on status of cloning e.g., dominant clone type in games. In this paper, we present an empirical study on code cloning in open source games by applying a state of the art clone detector, NiCad, and a clone visualization and analysis tool, VisCad. We identify both exact and near-miss clones from more than twenty open source C, Java and Python-based games, such as Jake2, Duo, Hexen2, jChess and OpenRPG from five different categories. Furthermore, we analyze a set of metrics for code clones in several different dimensions, including language, category, clone density and clone location to answer six essential research questions about the current status of cloning in open source games. Our research illustrates that cloning happens not only at inter-project level but also intra-project, specifically in First Person Shooter games developed using C (Type-1 50 percent clones). Such observation concretely shows the necessity of adopting clone management systems for game development.
\item[Web link:] \url{http://ieeexplore.ieee.org.ezproxy.falmouth.ac.uk/document/6901018/}
\item[Full text link:] \url{http://ieeexplore.ieee.org.ezproxy.falmouth.ac.uk/stamp/stamp.jsp?arnumber=6901018}
\item[Comments:] Write a few sentences on how you found the article and why you believe it is relevant and/or important.
\end{description}

\section*{Paper 3}
\begin{description}
\item[Title:] Work in progress Game development, social responsibility, and teaching
\item[Citation:] \cite{SocialTeaching}
\item[Abstract:] Video games are part of a powerful cultural industry that influences many people; therefore, educators and society cannot ignore that video games are here to stay. Professors should educate future developers to create games that entertain their audience and are socially responsible. This paper describes some assignments that can be used to teach social responsibility in game development.
\item[Web link:] \url{http://ieeexplore.ieee.org.ezproxy.falmouth.ac.uk/document/4417904/}
\item[Full text link:] \url{http://ieeexplore.ieee.org.ezproxy.falmouth.ac.uk/stamp/stamp.jsp?arnumber=4417904}
\item[Comments:] Write a few sentences on how you found the article and why you believe it is relevant and/or important.
\end{description}

\section*{Paper 4}
\begin{description}
\item[Title:] Study of Digital Video Watermarking
\item[Citation:] \cite{DigitalWatermarking}
\item[Abstract:] Digital watermarking, the technology of embedding special information into multimedia data, is a topic that has recently gained increasing attention all over the world. The watermark of digital images, audio, video, and other media products in general has been proposed for resolving copyright ownership and verifying the integrity of content. The paper first introduced the definition and basic framework of watermark techniques, and then the basic theoretical and evaluation criteria are expatiated. Finally, the application field and possible research direction of digital watermark technology is pointed out.
\item[Web link:] \url{http://ieeexplore.ieee.org.ezproxy.falmouth.ac.uk/document/6187968/}
\item[Full text link:] \url{http://ieeexplore.ieee.org.ezproxy.falmouth.ac.uk/stamp/stamp.jsp?arnumber=6187968}
\item[Comments:] Write a few sentences on how you found the article and why you believe it is relevant and/or important.
\end{description}

\section*{Paper 5}
\begin{description}
\item[Title:] Using chaotic 3D watermarking for game design copy right protection
\item[Citation:] \cite{Chaotic}
\item[Abstract:] Intellectual property will face further significant pressure to adapt in the coming years, as we make our way into the third decade of the commercial internet. Computer games deserve to be protected by copyright such as artwork, design docs, some user interface and program structure, sequence and organization, compilations of material, music, characters and story.
\item[Web link:] \url{http://ieeexplore.ieee.org.ezproxy.falmouth.ac.uk/document/6314579/}
\item[Full text link:] \url{http://ieeexplore.ieee.org.ezproxy.falmouth.ac.uk/stamp/stamp.jsp?arnumber=6314579}
\item[Comments:] Write a few sentences on how you found the article and why you believe it is relevant and/or important.
\end{description}

\section*{Paper 6}
\begin{description}
\item[Title:] ACM Code of Ethics and Professional Conduct
\item[Citation:] \cite{Code-of-Ethics}
\item[Abstract:] jv
\item[Web link:] \url{http://www.acm.org.ezproxy.falmouth.ac.uk/about-acm/acm-code-of-ethics-and-professional-conduct}
\item[Full text link:] \url{http://www.acm.org.ezproxy.falmouth.ac.uk/about-acm/acm-code-of-ethics-and-professional-conduct}
\item[Comments:] Write a few sentences on how you found the article and why you believe it is relevant and/or important.
\end{description}

\section*{Paper 7}
\begin{description}
\item[Title:] A Guided Tour of the Legal Implications of Software Cloning
\item[Citation:] \cite{SoftwareCloning}
\item[Abstract:] Software Cloning is the typical example where an interdisciplinary approach may bring additional elements into the community's discussion. In fact, little research has been done in its analysis from an Intellectual Propriety Rights (IPRs) perspective, even if it is a widely studied aspect of software engineering. An interdisciplinary approach is crucial to better understand the legal implications of software in the IPR context. Interestingly, the academic community of software and systems deals much more with such IPR issues than courts themselves. In this paper, we analyze some recent legal decisions in using software clones from a software engineering perspective. In particular, we survey the behavior of some major courts about cloning issues. As a major outcome of our research, it seems that legal fora do not have major concerns regarding copyright infringements in software cloning. The major contribution of this work is a case by case analysis of more than one hundred judgments by the US courts and the European Court of Justice. We compare the US and European courts case laws and discuss the impact of a recent European ruling. The US and EU contexts are quite different, since in the US software is patentable while in the EU it is not. Hence, European courts look more permissive regarding cloning, since “principles,” or “ideas,” are not copyrightable by themselves.
\item[Web link:] \url{http://ieeexplore.ieee.org.ezproxy.falmouth.ac.uk/document/7883344/}
\item[Full text link:] \url{http://ieeexplore.ieee.org.ezproxy.falmouth.ac.uk/stamp/stamp.jsp?arnumber=7883344}
\item[Comments:] Write a few sentences on how you found the article and why you believe it is relevant and/or important.
\end{description}

\section*{Paper 8}
\begin{description}
\item[Title:] 20 Years of research on intellectual property protection
\item[Citation:] \cite{20yearsIP}
\item[Abstract:] VLSI intellectual property (IP) reuse based design methodology was adopted by the semiconductor industry in the early 1990's and how to protect design IPs from piracy and misuse has since been a challenging problem. 2017 marks the 20th anniversary of the IP protection development and working group was founded and the first series of IP watermarking papers were published. In this paper, we survey the efforts from industry, government, and academia on securing the design IPs in the past 20 years with focus on development from academia side.
\item[Web link:] \url{http://ieeexplore.ieee.org.ezproxy.falmouth.ac.uk/document/8050602/}
\item[Full text link:] \url{http://ieeexplore.ieee.org.ezproxy.falmouth.ac.uk/stamp/stamp.jsp?arnumber=8050602}
\item[Comments:] Write a few sentences on how you found the article and why you believe it is relevant and/or important.
\end{description}

\section*{Paper 9}
\begin{description}
\item[Title:] The legal/juridical space of computer games: From intellectual property to intellectual freedom
\item[Citation:] \cite{LegalJuridical}
\item[Abstract:] Computer games may seem mere entertainments, but they are a multi-billion dollar global business of multi-media stories that, beyond entertainment, teach, edify and offend. As with their predecessors - stories, poetry, song, books, cinema and video - the art, politics and business of their creation and commerce have substantial impact in the modern world. In response, the laws of modernity impact computer games. We examine how computer gaming is engaged in the transnational legal regime of intellectual property, personal and state rights and intellectual freedom.
\item[Web link:] \url{http://ieeexplore.ieee.org.ezproxy.falmouth.ac.uk/document/6934141/}
\item[Full text link:] \url{http://ieeexplore.ieee.org.ezproxy.falmouth.ac.uk/stamp/stamp.jsp?arnumber=6934141}
\item[Comments:] Write a few sentences on how you found the article and why you believe it is relevant and/or important.
\end{description}

\section*{Paper 10}
\begin{description}
\item[Title:] Justifying legal protection of intellectual property: the interests argument
\item[Citation:] \cite{LegalProtection}
\item[Abstract:] Whether or not intellectual property rights ought, as a matter of political morality, to be protected by the law surely depends on what kinds of interests the various parties have in intellectual content. Although theorists disagree on the limits of morally legitimate lawmaking authority, this much seems obvious: the coercive power of the law should be employed only to protect interests that rise to a certain level of moral importance. We have such a significant interest in not being lied to, for example, that ordinary unilateral lies are morally wrong, but the wrongness of lying does not rise to the level of something the state should protect against by coercive criminal prohibition. Indeed, it would clearly be wrong for the law to coercively restrict behaviors in which no one has any morally significant interests (i.e., interests that are important enough from the standpoint of morality that they receive some protection from moral principles) whatsoever. Using the coercive power of the law to restrict freedom is not justified unless the moral benefits of restricting the behavior outweigh the moral costs involved in using force to restrict human autonomy and freedom.
\item[Web link:] \url{https://dl-acm-org.ezproxy.falmouth.ac.uk/citation.cfm?id=1497055&CFID=993001348&CFTOKEN=67703433}
\item[Full text link:] \url{http://delivery.acm.org.ezproxy.falmouth.ac.uk/10.1145/1500000/1497055/p13-himma.pdf?ip=193.61.64.8&id=1497055&acc=ACTIVE%20SERVICE&key=BF07A2EE685417C5%2EEAA225A8AB01C582%2E4D4702B0C3E38B35%2E4D4702B0C3E38B35&CFID=993001348&CFTOKEN=67703433&__acm__=1507479041_12ef7f4673b27ce9d47685c0a10f41aa}
\item[Comments:] Write a few sentences on how you found the article and why you believe it is relevant and/or important.
\end{description}

\section*{Paper 11}
\begin{description}
\item[Title:] Thou shalt not…A look at the ethics of copying software code
\item[Citation:] \cite{ThouShaltNot}
\item[Abstract:] Since the 1970's, the field of ethics in software engineering has attempted to define the boundaries of what was morally correct when dealing with problems aggravated, transformed, or created by computer technology. Efforts to codify ethics for computer software engineers resulted in bright line rules such as “thou shalt not appropriate other people's intellectual output” [1] and “honor property rights including copyrights and patent” [2]. Few instances in practice are, however, as black and white as these rules suggest. Rather, there are a number of grey areas where computer software engineers must question whether an action is morally correct. One ambiguity is when and to what extent it is morally acceptable to copy computer software code. This paper investigated whether software engineers comply with existing ethical standards surrounding intellectual property rights associated with computer software code.
\item[Web link:] \url{http://ieeexplore.ieee.org.ezproxy.falmouth.ac.uk/document/6893375/}
\item[Full text link:] \url{http://ieeexplore.ieee.org.ezproxy.falmouth.ac.uk/stamp/stamp.jsp?arnumber=6893375}
\item[Comments:] Write a few sentences on how you found the article and why you believe it is relevant and/or important.
\end{description}

\section*{Paper 12}
\begin{description}
\item[Title:] Unauthorized copying of software: an empirical study of reasons for and against
\item[Citation:] \cite{Unauthorized}
\item[Abstract:] Computer users copy computer software - this is well-known. However, less well-known are the reasons why some computer users choose to make unauthorized copies of computer software. Furthermore, the relationship linking the theory and the practice is unknown, i.e., how the attitudes of ordinary end-users correspond with the theoretical views of computer ethics scholars. In order to fill this gap in the literature, we investigated the moral attitudes of 249 Finnish computing students towards the unauthorized copying of computer software, and we then asked how these results compared with the theoretical reasons offered by computer ethics scholars. The results shed a new light on students' moral attitudes with respect to the unauthorized copying of software. In particular, this new knowledge is useful for computer ethics teachers, and for organizations seeking to combat this practice.
\item[Web link:] \url{https://dl-acm-org.ezproxy.falmouth.ac.uk/citation.cfm?id=1273357&CFID=993665928&CFTOKEN=23987971}
\item[Full text link:] \url{http://delivery.acm.org.ezproxy.falmouth.ac.uk/10.1145/1280000/1273357/p30-siponen.pdf?ip=193.61.64.8&id=1273357&acc=ACTIVE%20SERVICE&key=BF07A2EE685417C5%2EEAA225A8AB01C582%2E4D4702B0C3E38B35%2E4D4702B0C3E38B35&CFID=993665928&CFTOKEN=23987971&__acm__=1507660068_578847db7b6ff5cbaaaeb952a135f190}
\item[Comments:] Write a few sentences on how you found the article and why you believe it is relevant and/or important.
\end{description}

\bibliographystyle{IEEEtran}
\bibliography{initial_references}

\end{document}
