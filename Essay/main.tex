% Please do not change the document class
\documentclass{scrartcl}

% Please do not change these packages
\usepackage[hidelinks]{hyperref}
\usepackage[none]{hyphenat}
\usepackage{setspace}
\doublespace

% You may add additional packages here
\usepackage{amsmath}

% Please include a clear, concise, and descriptive title
\title{What are the ethical issues of cloning video games and what can be done to prevent the copying of Intellectual Property?}

% Please do not change the subtitle
\subtitle{COMP230 - Ethics and Professionalism}

% Please put your student number in the author field
\author{1603748}

\begin{document}

\maketitle

\abstract{The games industry Is an extremely competitive market, and it is often a challenge for developers to think of good game ideas, this can lead to them "stealing" the idea from other developers. This is known as "cloning", they take the core idea of the game, sometimes they change it a little bit and change the look of the game, then they sell it as if it were their own game. In this essay I will talk about the ethical and moral issues that this causes and potential solutions to prevent cloning from happening. Is it wrong to clone a game, should it remain legal as long as you change the aesthetic of the game, or should there be stricter laws introduced to prevent the copy of Intellectual Property.}

\section*{Introduction}
 
In this essay I will discuss the ethical and moral issues of video game cloning and talk about the Intellectual Property (IP) laws that protect your work and ideas. I will talk about different clone types, for example, an exact clone includes program fragments which are identical whereas a semantic clone is functionally similar but are not formally identical. I will also argue whether the cloning of a game is harmful or not to the owner, some consider it not harmful because the owner still has the original game, others argue it is harmful as consumers may buy the cloned game rather than the original meaning the owner loses money. The current IP laws usually do not provide adequate protection against cloning, I will discuss the different IP laws and what they can protect in a video game context in the first section.

\section*{Intellectual Property Protection}

In order to protect software from being cloned the creators must obtain legal protection. There are various intellectual property rights including patents, copyright, design right, and trade secret. For software having only one of these protections Is not always adequate protection, ``patent protection can play a limited role in the legal protection of computer programs, but does not provide an adequate solution for the basic legal protection of such works. When at all possible, both patent and copyright protection should be pursued.'' \cite{Current}. Having protection such as copyright and patent are a must whenever possible as it remains the only legal way to get back the revenue lost from IP piracy \cite{20yearsIP}. However it has been argued that if a program qualifies for patent protection, it cannot also qualify for copyright protection \cite{Current}. Copyright is limited in what it is able to cover, for example in games it covers the artistic expressions like the aesthetic, music and the storyline, while a patent can protect the mechanisms used to create a copyright-able object \cite{LegalJuridical}. A patent is also quite difficult to obtain for a video game as ``To obtain a patent, the computer software
must be new, useful, and non-obvious'' \cite{ThouShaltNot}. The game mechanics, what makes a game actually a game, are not protected by copyright or patent protection \cite{ClonedAtBirth}, I believe this should be changed. I believe with some sort of protection against copying core game mechanics, the cloning of games could be significantly reduced. Without any legal protection, there is nothing stopping a developer from taking all the mechanics from a previous game and using it their own game and making money from it. 

\section*{Ethics of Cloning}

The field of ethics on software engineering has outlined what they believe to be morally right by making rules that software engineers are obliged to follow. For example ``Computing professionals are obligated to protect the integrity of intellectual property. Specifically, one must not take credit for other's ideas or work, even in cases where the work has not been explicitly protected by copyright, patent, etc.'' \cite{Code-of-Ethics} and ``Thou Shalt Not Appropriate Other People’s Intellectual Output''. As the popularity of video games has grown, the competition in the industry has also grown \cite{adaptation}, which has led to game developers taking a bit too much inspiration from other games, ``everyone is borrowing of everyone in this industry'' \cite{ClashOfClones}. An ethical concern is the fact that borrowing ideas is usually accepted because it is often hard to tell the difference between 'copying' and 'being inspired by' \cite{GameDesignIdeas}. This means a developer can copy an idea from another game and simply pass it off as being inspired not copied. Furthermore these "clone developers" will potentially capitalize on the success of the existing idea and use it for their own profit \cite{AttackOfTheClones}. On the other hand it can be argued that taking inspiration from other games is a good thing, Isaac Newton said "If I have seen further it is by standing on the shoulders of Giants", in the games industry developers take games that have already been made and improve them. In theory this means that games will get better and better over time because game ideas will be improved on, and then those ideas will be further improved and refined. It is possible that developers attempt to generate new ideas of their own but it is often very challenging \cite{Creativity} so they take inspiration from existing games. If someone uses an idea from another popular game it is effectively taking a shortcut to a game that is proven to be highly profitable. In the video games industry you could argue that there are multiple types of clones, similar to how there are different software clone types \cite{SoftCloning,PlagiarismDetect}. An exact clone would be a game that has directly copied a popular and trending game to make money with minimal effort. This is extremely popular in mobile games \cite{tatap,zoomIOS}, some games are so popular they have hundreds of clones, such as Flappy Bird. This is unethical because multiple over developers are benefiting from the original owners work, however It could be argued that it does not harm the owner because they still own the original game, despite the fact that others own clones of it. This is similar to the argument that making unauthorised copies of software does not harm the owner of the software \cite{Unauthorized}.

\section*{Conclusion}
This essay has summarised the current IP laws and shown that they tend to provide an inadequate protection for video games, they can protect the copy of assets but not ideas and mechanics. I feel that these laws should be refined to protect the ideas and concepts of games to minimise the number of clones on the market. The ethics of cloning and taking inspiration was also discussed, I conclude from this that it is unethical and morally wrong to make exact clones of games, however if a game merely takes inspiration from an existing game then it can be beneficial and ethical.


\bibliographystyle{IEEEtran}
\bibliography{references}


\end{document}
